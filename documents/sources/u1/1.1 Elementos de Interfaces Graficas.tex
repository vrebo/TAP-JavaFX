\documentclass[12pt,letterpaper]{book}

\usepackage{amsmath}
\usepackage{amsfonts}
\usepackage{amssymb}
\usepackage{booktabs}
\usepackage[spanish,es-tabla]{babel}
\usepackage[utf8]{inputenc}
\usepackage{fancyhdr}
\usepackage{framed}
\usepackage{geometry}
\usepackage{titlesec}
\usepackage{caption}
\usepackage{graphicx}
\usepackage{listings}
\usepackage{xcolor}
\usepackage{courier}

\definecolor{light-gray}{HTML}{FFFFFF}
\definecolor{java-comment}{rgb}{0,0.5,0}
\definecolor{java-keyword}{rgb}{0.13,0.13,1}
\definecolor{java-literal}{rgb}{0,0.6,0}
\definecolor{java-annotation}{rgb}{0.46,0.45,0.48}
\definecolor{java-string}{HTML}{B36B00}

%	Ubica el pie de tabla en la parte inferior
\captionsetup[table]{position=bottom}
%	Cambia los margenes del documento
\geometry{left=25mm,top=20mm,right=25mm}
%	Modifica el formato del comando \section
\titleformat{\section}{\large\bfseries\raggedleft}{Lección \thesection}{1em}{}[{\titlerule[0.8pt]}]
%	Modifica el comando \chapter
\addto\captionsspanish{\renewcommand{\chaptername}{Unidad}}

%	Modifica el encabezado y pie de página
\pagestyle{fancy}
\fancyhf{}
\rhead{\chaptername : Elementos de Interfaces Gráficas}
\lhead{Tópicos Avanzados de Programación}
\rfoot{Página \thepage}
\lfoot{Rafael Rivera López}

%	Agrega espaciado en el interior del las filas de las tablas
\renewcommand{\arraystretch}{1.5}

%Para cambiar el grosor de la línea de encabezado o pie de página
%\renewcommand{\headrulewidth}{2pt}
\renewcommand{\footrulewidth}{0.5pt}

\renewcommand{\lstlistingname}{Código}

\lstset{
  aboveskip=3mm,
  belowskip=3mm,
  literate=
  {á}{{\'a}}1 {é}{{\'e}}1 {í}{{\'i}}1 {ó}{{\'o}}1 {ú}{{\'u}}1
  {Á}{{\'A}}1 {É}{{\'E}}1 {Í}{{\'I}}1 {Ó}{{\'O}}1 {Ú}{{\'U}}1
  {à}{{\`a}}1 {è}{{\`e}}1 {ì}{{\`i}}1 {ò}{{\`o}}1 {ù}{{\`u}}1
  {À}{{\`A}}1 {È}{{\'E}}1 {Ì}{{\`I}}1 {Ò}{{\`O}}1 {Ù}{{\`U}}1
  {ä}{{\"a}}1 {ë}{{\"e}}1 {ï}{{\"i}}1 {ö}{{\"o}}1 {ü}{{\"u}}1
  {Ä}{{\"A}}1 {Ë}{{\"E}}1 {Ï}{{\"I}}1 {Ö}{{\"O}}1 {Ü}{{\"U}}1
  {â}{{\^a}}1 {ê}{{\^e}}1 {î}{{\^i}}1 {ô}{{\^o}}1 {û}{{\^u}}1
  {Â}{{\^A}}1 {Ê}{{\^E}}1 {Î}{{\^I}}1 {Ô}{{\^O}}1 {Û}{{\^U}}1
  {œ}{{\oe}}1 {Œ}{{\OE}}1 {æ}{{\ae}}1 {Æ}{{\AE}}1 {ß}{{\ss}}1
  {ű}{{\H{u}}}1 {Ű}{{\H{U}}}1 {ő}{{\H{o}}}1 {Ő}{{\H{O}}}1
  {ç}{{\c c}}1 {Ç}{{\c C}}1 {ø}{{\o}}1 {å}{{\r a}}1 {Å}{{\r A}}1
  {€}{{\euro}}1 {£}{{\pounds}}1 {«}{{\guillemotleft}}1
  {»}{{\guillemotright}}1 {ñ}{{\~n}}1 {Ñ}{{\~N}}1 {¿}{{?`}}1,
  breaklines=true,
  breakatwhitespace=true,
  postbreak=\raisebox{0ex}[0ex][0ex]{\ensuremath{\color{red}\hookrightarrow\space}},
  language=Java,
  framesep=5pt,
  frame=Trbl,
  basicstyle=\scriptsize\ttfamily,
  columns=fullflexible,
  backgroundcolor=\color{light-gray},
  commentstyle=\color{java-comment},
  keywordstyle=\bfseries\color{java-keyword},
  stringstyle=\color{java-string},
  morecomment=[s][\color{gray}]{@}{\ }
}

\fboxsep=6mm 	%Añade un espacio enter el marco y el contenido de un \fbox{}
\fboxrule=1pt 	%Modifica el grosor del marco de \fbox{}

\begin{document}
\chapter{Elementos de Interfaces Gráficas}
\section{Elementos de una Interfaz Gráfica}

Una Interfaz Gráfica de Usuario (GUI, Graphical User Interface) es el conjunto de componentes gráficos que posibilitan la interacción entre el usuario y la aplicación: ventanas, botones, listas, listas desplegables, cajas de diálogo, campos de texto, etc.\\

Desde el punto de vista de Java, toda interfaz gráfica tiene:

\begin{itemize} \itemsep0em
\item \textbf{Componentes:} Objetos que representan un elemento gráfico (botón, cuadro de texto).
\item \textbf{Contenedores:} Objetos que contienen componentes (Ventanas, Cuadros de Dialogo).
\item \textbf{Administradores de diseño:} Objetos que controlan la forma en que se ubican los componentes dentro de un contenedor.
\item \textbf{Contexto Gráfico:} Área de un componente gráfico donde se pueden dibujar primitivas gráficas (líneas, curvas) y colocar imágenes.
\item \textbf{Eventos:} Objetos que representan un cambio en un componente, generalmente producido por el usuario al realizar alguna operación.
\end{itemize}

\subsection{Componentes gráficos}
 
Un componente es un objeto que tiene una representación gráfica que puede ser presentada en una pantalla y puede interactuar con el usuario. La clase abstracta Control contiene los métodos comunes para todos los componentes gráficos.\\

\begin{table}[!hbtp] \footnotesize \centering
\caption{Métodos más utilizados en un componente gráfico}
	\begin{tabular}{l  l} \toprule[1.5pt]
	\textbf{Método} 				& \textbf{Descripción}\\ \midrule[1.5pt]
	setBackground(Background b) 	& Define fondo, color o imagen, del componente. \\ \hline
	setCursor(Cursor c) 			& Define el cursor a mostrar sobre el componente. \\ \hline
	setBorder(Border b) 			& Define los bordes del componente.\\ \bottomrule[1.5pt]
	\end{tabular}
\end{table}


Los desarrolladores cuentan con tres alternativas de herramientas para construir interfaces gráficas en Java. La más antigua, AWT; la más utilizada y con el mayor tiempo en la preferencia de los desarrolladores, SWING; y la más reciente, JavaFX, que tiene cambios importantes en el paradigma de construcción de interfaces.\\

A continuación se presentan componentes de las tres alternativas.

\begin{figure}[!htbp] \centering \fboxsep=1mm
\fbox{\includegraphics[scale=0.9]{images/componentes-awt}}
\caption{Algunos componentes de AWT}
\end{figure}
 
\begin{figure}[!htbp] \centering \fboxsep=1mm
\fbox{\includegraphics[scale=0.8]{images/componentes-swing}}
\caption{Algunos componentes de Swing}
\end{figure}

\begin{figure}[!htbp] \centering \fboxsep=1mm
\fbox{\includegraphics[scale=0.8]{images/componentes-jfx}}
\caption{Algunos componentes de JavaFX}
\end{figure}

\subsection{Contenedores gráficos}

Un contenedor es un componente gráfico que tiene la capacidad de almacenar otros componentes gráficos.

\begin{table}[!hbtp] \footnotesize \centering
	\caption{Métodos más utilizados en contenedores}
	\begin{tabular}{l l} \toprule[1.5pt]
	\textbf{Método} 		& \textbf{Descripción}\\ \midrule[1.5pt]
	setScene(Scene value) 	& Define la escena que se mostrará en el contenedor. \\ \hline
	show() 					& Hace visible el contenedor. \\ \hline
	sizeToScene() 			& Redimensiona el contenedor al tamaño de su contenido.\\ \bottomrule[1.5pt]
	\end{tabular}
\end{table}


\subsection{Administradores de diseño}

Un administrador de diseño (layout) es un objeto que se encarga de controlar la ubicación y distribución de los componentes dentro de un contenedor. Un layout se considera otro nodo dentro del grafo de escena de una interfaz gráfica, y más que modificar el comportamiento de los contenedores principales son contenedores en sí.

\newpage
\subsection{Diseño de una interfaz gráfica}

Una interfaz gráfica común tiene los siguientes elementos:

\begin{figure}[!htbp] \centering \fboxsep=1mm
\fbox{\includegraphics[scale=0.6]{images/interfaz-tipica}}
\caption{Elementos comunes de una interfaz gráfica de usuario}
\end{figure}

El diseño de una interfaz gráfica debe ser jerárquico. La ventana estará construida por los siguientes elementos:

\begin{itemize} \itemsep0em
\item Una barra de menú (opcional).
\item Un contenedor gráfico para los componentes que definen la interfaz gráfica.
\item Un objeto que maneje los eventos de la interfaz.
\end{itemize}

\begin{figure}[!htbp] \centering 
\fbox{\includegraphics[scale=0.47]{images/gui-general}}
\caption{Estructura general de una interfaz gráfica de usuario}
\end{figure}

\subsubsection{Barra de Menú}

Es un área de la interfaz de usuario que indica y presenta las opciones o herramientas de una aplicación dispuestas en menús desplegables.
En la mayoría de entornos de escritorio los diferentes menús presentes en estas barras pueden ser desplegados por medio de atajos de teclado, al mantener presionada la tecla ALT y la tecla correspondiente a la letra subrayada en la barra de menú.

\begin{figure}[!htbp] \centering \fboxsep=1mm
\fbox{\includegraphics[scale=0.8]{images/barra-menu}}
\caption{Ejemplo de una barra de menú y varias opciones de menú}
\end{figure}

\subsubsection{Contenedor gráfico}

Define el área donde se construirá la interfaz gráfica. Este contenedor sirve como organizador de los componentes gráficos.

\begin{figure}[!htbp] \centering \fboxsep=1mm
\fbox{\includegraphics[scale=0.9]{images/contenedor}}
\caption{Ejemplo de un contenedor con varios componentes gráficos}
\end{figure}

\subsubsection{Manejador de Eventos}

Cuando un usuario utiliza una interfaz gráfica, realiza cambios sobre los componentes (pulsa un botón, captura sobre un campo, etc). Estos cambios se reflejan en objetos denominados eventos.

Los eventos deben ser atendidos para que la interfaz tenga el comportamiento esperado por el usuario, por lo cual toda interfaz gráfica debe utilizar un elemento manejador de eventos.


\begin{figure}[!htbp] \centering \fboxsep=1mm
\fbox{\includegraphics[scale=0.8]{images/eventos}}
\caption{Ejemplo de un manejo de evento. Al oprimir el botón, se despliega una ventana de diálogo}
\end{figure}

\subsubsection{Estructura de clases para una interfaz gráfica}

En este curso, cada aplicación gráfica estará diseñada dentro de un paquete, y tendrá una clase aplicación (donde se define el método main), y una o varias clases instanciables que definen los menús, los contenedores y los demás elementos necesarios para la interfaz.\\

La clase aplicación creará una instancia del \textbf{contenedor principal} (la ventana), definirá su configuración (tamaño, ubicación), le asignará la barra de menú correspondiente (si esta existe) y el contenedor para los componentes gráficos. Esta clase también creará los objetos que atenderán los eventos producidos en la aplicación para el comportamiento adecuado de esta.\\

\begin{lstlisting}[caption=Creación de una ventana simple]  % Start your code-block

import javafx.application.Application;
import javafx.scene.Parent;
import javafx.scene.Scene;
import javafx.stage.Stage;


public class Ventana extends Application {

    @Override
    public void start(Stage primaryStage) {
        Parent root = new Pane();
        Scene scene = new Scene(root);
        
        primaryStage.setTitle("Ejemplo de Ventana");
        primaryStage.setScene(scene);
        primaryStage.setWidth(400);
        primaryStage.setHeight(300);
        primaryStage.show();
    }

    public static void main(String[] args) {
        launch(args);
    }
}
\end{lstlisting}

\end{document}