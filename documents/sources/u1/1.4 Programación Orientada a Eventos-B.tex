\documentclass[12pt,letterpaper]{book}

\usepackage[export]{adjustbox}
\usepackage{amsmath}
\usepackage{amsfonts}
\usepackage{amssymb}
\usepackage[spanish,es-tabla]{babel}
\usepackage{booktabs}
\usepackage{caption}
\usepackage{fancyhdr}
\usepackage{framed}
\usepackage{geometry}
\usepackage{graphicx}
\usepackage[utf8]{inputenc}
\usepackage{listings}
\usepackage{titlesec}
\usepackage{xcolor}
\usepackage{courier}

\usepackage{etoolbox}

\definecolor{light-gray}{HTML}{FFFFFF}
\definecolor{java-comment}{rgb}{0,0.5,0}
\definecolor{java-keyword}{rgb}{0.13,0.13,1}
\definecolor{java-literal}{rgb}{0,0.6,0}
\definecolor{java-annotation}{rgb}{0.46,0.45,0.48}
\definecolor{java-string}{HTML}{CE7B00}
\definecolor{java-const}{HTML}{6600CC}


%	Cambia los margenes del documento
\geometry{left=23mm,top=20mm,right=23mm}
%	Modifica el formato del comando \section
\titleformat{\section}{\large\bfseries\raggedleft}{Lección \thesection}{1em}{}[{\titlerule[0.8pt]}]
%	Modifica el comando \chapter
\addto\captionsspanish{\renewcommand{\chaptername}{Unidad}}

%	Modifica el encabezado y pie de página
\pagestyle{fancy}
\fancyhf{}
\rhead{\chaptername : Elementos de Interfaces Gráficas}
\lhead{Tópicos Avanzados de Programación}
\rfoot{Página \thepage}
\lfoot{Rafael Rivera López}

%	Agrega espaciado en el interior del las filas de las tablas
\renewcommand{\arraystretch}{1.5}

%Para cambiar el grosor de la línea de encabezado o pie de página
%\renewcommand{\headrulewidth}{2pt}
\renewcommand{\footrulewidth}{0.5pt}

\renewcommand{\lstlistingname}{Código}



\lstset{
  aboveskip=3mm,
  belowskip=3mm,
  literate=
  {á}{{\'a}}1 {é}{{\'e}}1 {í}{{\'i}}1 {ó}{{\'o}}1 {ú}{{\'u}}1
  {Á}{{\'A}}1 {É}{{\'E}}1 {Í}{{\'I}}1 {Ó}{{\'O}}1 {Ú}{{\'U}}1
  {à}{{\`a}}1 {è}{{\`e}}1 {ì}{{\`i}}1 {ò}{{\`o}}1 {ù}{{\`u}}1
  {À}{{\`A}}1 {È}{{\'E}}1 {Ì}{{\`I}}1 {Ò}{{\`O}}1 {Ù}{{\`U}}1
  {ä}{{\"a}}1 {ë}{{\"e}}1 {ï}{{\"i}}1 {ö}{{\"o}}1 {ü}{{\"u}}1
  {Ä}{{\"A}}1 {Ë}{{\"E}}1 {Ï}{{\"I}}1 {Ö}{{\"O}}1 {Ü}{{\"U}}1
  {â}{{\^a}}1 {ê}{{\^e}}1 {î}{{\^i}}1 {ô}{{\^o}}1 {û}{{\^u}}1
  {Â}{{\^A}}1 {Ê}{{\^E}}1 {Î}{{\^I}}1 {Ô}{{\^O}}1 {Û}{{\^U}}1
  {œ}{{\oe}}1 {Œ}{{\OE}}1 {æ}{{\ae}}1 {Æ}{{\AE}}1 {ß}{{\ss}}1
  {ű}{{\H{u}}}1 {Ű}{{\H{U}}}1 {ő}{{\H{o}}}1 {Ő}{{\H{O}}}1
  {ç}{{\c c}}1 {Ç}{{\c C}}1 {ø}{{\o}}1 {å}{{\r a}}1 {Å}{{\r A}}1
  {€}{{\euro}}1 {£}{{\pounds}}1 {«}{{\guillemotleft}}1
  {»}{{\guillemotright}}1 {ñ}{{\~n}}1 {Ñ}{{\~N}}1 {¿}{{?`}}1,
  breaklines=true,
  breakatwhitespace=true,
  postbreak=\raisebox{0ex}[0ex][0ex]{\ensuremath{\color{red}\hookrightarrow\space}},
  language=Java,
  framesep=5pt,
  frame=Trbl,
  basicstyle=\scriptsize\ttfamily,
  columns=fullflexible,
  backgroundcolor=\color{light-gray},
  commentstyle=\color{java-comment},
  keywordstyle=\bfseries\color{java-keyword},
  stringstyle=\color{java-string},
  morecomment=[s][\color{gray}]{@}{\ },
  moredelim={[is][\bf]{\#bf}{\#bf}},
}

\fboxsep=6mm 	%Añade un espacio enter el marco y el contenido de un \fbox{}
\fboxrule=1pt 	%Modifica el grosor del marco de \fbox{}
 
 

\begin{document}
\chapter{Elementos de Interfaces Gráficas}
\section{Programación Orientada a Eventos}

\subsection{Clases Internas Anónimas}


\subsubsection{Relación de los Constructores y la Herencia}

\begin{itemize}
\item Cuando una clase (subclase) hereda de otra (superclase), se entiende que la subclase contiene todos los atributos y métodos que están definidos en la superclase.
\item El constructor de la subclase siempre invoca al constructor de la superclase, para inicializar los atributos definidos en la superclase.
\end{itemize}

\begin{minipage}[c]{0.95\textwidth}
\lstinputlisting[
	caption=Clase padre.
]{source-codes/t-1-4_code-4_Father.java}
\end{minipage}

\begin{minipage}[c]{0.95\textwidth}
\lstinputlisting[
	caption=Clase hijo que hereda de la clase Father.
]{source-codes/t-1-4_code-4_Son.java}
\end{minipage}

\begin{minipage}[c]{0.95\textwidth}
\lstinputlisting[
	caption=Clase main de la aplicación de ejemplo de herencia.
]{source-codes/t-1-4_code-4_Main.java}
\end{minipage}

Al compilar, el constructor de la subclase invoca al constructor de la superclase:

\begin{minipage}[c]{0.95\textwidth}
\lstinputlisting[
	caption=Explicación del funcionamiento de herencia y constructores.
]{source-codes/t-1-4_code-4_Son_explained.java}
\end{minipage}

\subsubsection{Ejemplo con Paneles}

\begin{figure}[!h] \centering \fboxsep=1mm 
\fbox{\includegraphics[width=0.9\textwidth]{images/lorem-ipsum}}
\caption{Aplicación con tres paneles, un padre, un hijo y un nieto}
\end{figure}

\begin{minipage}[c]{0.95\textwidth}
\lstinputlisting[
	caption=Clase PolygonPane.
]{source-codes/t-1-3_code-6_PolygonPane.java}
\end{minipage}

\end{document}
